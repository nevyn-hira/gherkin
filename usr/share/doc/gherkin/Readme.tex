\documentclass{article}
\usepackage{times}
\begin{document}
\title{Gherkin}
\section{Introduction}
\setcounter{secnumdepth}{5}
Gherkin is a piece of software to allow for provisioning a computer on first
time boot. Essentially, it allows the distributor to ask the user a couple of
questions and to set up a machine accordingly.\\
It's original intent was to set up a computer depending on site. i.e.
a computer could be customized to a particular school, building, department
etc. but this would also work equally well for software selection.\\
It's main power is in what it allows the distributor to do in that
the distributor can write what is essentially a really simple "definition"
file to tell it what to do and how to do it.
\section{Deployment}
Thus far gherkin has only been used on a deb based system and currently
doesn't accomodate other packaging systems though it should be relatively
trivial to make it do so.\\
To use gherkin, you need to alter:\\
\texttt{/usr/local/share/gherkin/constants.py}
to suit your deployment.\\
Sample:
\begin{verbatim}
#!/usr/bin/env python

#CANCEL_CODE provides a 'code' for canceled dialogues.
#It needs to be a value unlikely to ever be entered. By having it here, it allows
#the UI components to use this constant as well as gherkin to use the same code to
#test for a canceled diaglogue.
CANCEL_CODE = "b748d7f9f3d5197c46052_120625a4550131d8 67049c13e11,02d.7?sa83a2e39b4ff"

#Change these to individual need.
#The title is probably a good place to have some sort of title as a grouping.
SITE_SELECTION_DIALOGUE_TITLE = "Manaiakalani Schools"
SITE_SELECTION_DIALOGUE_MESSAGE = "Select your school"
SITE_SHOW_TEXT = "School"

#Folder for definiton files
DEFINITIONS_FOLDER = "/usr/local/share/gherkin/definitions"

#Default Groups (when Groups not specified when adding users)
DEFAULT_GROUPS="dialout,cdrom,dip,plugdev"

#Don't change these. Bunch of constants that are used internally
LEFT = 0
RIGHT = 1
TOP = 0
BOTTOM = 1
\end{verbatim}
Follow the comments in the file. The ones of particular interest are:
\begin{itemize}
  \item SITE\_SELECTION\_DIALOGUE\_TITLE
  \item SITE\_SELECTION\_DIALOGUE\_MESSAGE
  \item SITE\_SHOW\_TEXT
\end{itemize}
You may also be interested in:
\begin{itemize}
  \item DEFAULT\_GROUPS
  \item DEFINITIIONS\_FOLDER
\end{itemize}
You will need to use an xsession.desktop file. One has been included in:
\begin{verbatim}
/usr/share/xsessions
\end{verbatim}
The purpose of this file is to run gherkin, and only gherkin within
an x-session. Running only gherkin leads to less complications i.e.
the Window is less likely to disappear (though it's still possible).
\subsection{Pre-Configuration}
In the standard source tree you will find a file called \texttt{constants.py} in:\\
\texttt{/usr/local/share/gherkin}
\section{Definition Files}
The definition files are designed to be really simple. The complexity is
more in making up the packages for gherkin to use. However, making packages
is outside of the scope of this documentation. It is my hope that such
information can be put into a wiki at some stage.
\subsection{Definition File}
\subsubsection{Header}
\paragraph{site\_name}
Syntax:\\
site\_name = [site name]\\
Use this to set how this entity appears in the site selection screen. i.e.\\
\texttt{site\_name = Main Building}
\subsection{Get}
\subsubsection{}
\end{document}
